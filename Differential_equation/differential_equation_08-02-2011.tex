\section{Линейное однородное уравнение с переменными коэффициентами.}
В этой главе мы будем изучать свойства решений линейных уравнений 2го порядка
$$x'' + p(t)x' + q(t)x = 0, t \in (a, b) \eqno{6.1}$$
ниже предполагается, что $p(t) \in C^1 (a, b), q(t) \in C(a, b)$

\subsection{Теоремы Штурма}

\begin{lemma}
	Пусть $x(t), t \in (a,b)$ - решение уравнения (6.1);
	Тогда мн-во его нулей на (содержащихся в $(a,b)$) любых отрезках $[t_1,t_2] \subset (a,b)$ конечно.
	
	\proof
	Если $x(t), t \in (a,b)$ - решение уравнения (6.1), то $x(t) \in C^1(a,b)$ \\
	Предположим, противное, т.е. решение $x(t)$ имеет бесконечное число нулей на $[t_1,t_2]$. \\
	Выберем сходящуюся посл-ть нулей $\{t_k\},t_k \to t_0, k \to\infty$ 
	В силу непрерывности решения $x(t)$ мы имеем
	$\lim_{k\to\infty}x(t_k)= x(t_0)=0$, т.к. $\forall k \to x(t_k)=0$
	$$x'(t_0)=\lim_{k\to\infty}\frac{x(t)-x(t_0)}{t - t_0} = \lim_{t_k \to t_0}\frac{x(t_k) - x(t_0)}{t_k - t_0}=0$$
	$$x(t_0)=x'(t_0)=0$$
	По теореме о существовании и единственности решения ЗК - единственное решение
	$x(t)\equiv 0, t \in [t_1,t_2]$ - противоречие.
\end{lemma}
\qed 

\begin{corollary}
	Если нетривиальное решение $x(t) \in (a,b)$ имеет конечное число нулей, то их можно перенумеровать.
	
	\proof
	Назовем $t_1,t_2$ последовательными нулями решения $x(t)$, если $x(t)$ не имеет нулей на $(t_1,t_2)$\\
	Произведем замену функции $x(t)$  в уравнении (6.1) на ф-ию $y(t)$ согласно формуле
	$$x(t) = y(t)e^{-1/2 \int_{t_0}^t p(\tau)d\tau} \eqno{6.2}$$
	В результате подстановки (6.2) получим уравнение для $y(t)$:
	$$y'' + Q(t)y(t) = 0, t \in (a,b) \eqno{6.3}$$
	$$Q(t) = q(t) - \frac{p^{2}(t)}{4} - \frac{p'(t)}{2} \in C(a,b)$$
	Рассмотрим уравнение
	$$y'' + Q_1(t)y = 0 \eqno{6.4}$$
	$$z'' + Q_2(t)z = 0 \eqno{6.5}$$
	$$Q_2(t),Q_1(t)\in C(a,b),t\in(a,b)$$
\end{corollary}
\qed

\begin{theorem}[<Штурма о сравнении>]
	Пусть $Q_1(t) \le Q_2(t), t\in(a,b)$, $t_1,t_2$ - два последовательных нуля решения $y(t)$ уравнения (6.4).\\
	Тогда любое решение $z(t)$  уравнения (6.5) имеет хотя бы 1 ноль на $[t_1,t_2]$.
	
	\proof
	По условию Th. решение $y(t)$ уравнения (6.4) на $(t_1,t_2)$ не меняет знак.\\
	Предположим обратное: решение $z(t)$ уравнения (6.5) не имеет нулей на $[t_1,t_2]$ и будем считать, что
	$z(t) > 0, t\in[t1,t2]$\\
	Умножим уравнение (6.4) на $z(t)$, а уравнение  (6.5) на $y(t)$ и вычтем одно из другого, тогда 
	$$zy'' - yz'' = (Q_2(t) - Q_1(t))yz \eqno{6.6}$$
	Нетрудно видеть, что левая часть (6.6) преставима в виде $$zy'' - yz'' = d(zy'-yz')$$
	Тогда $$d/dt(zy'-yz')=(Q_2-Q_1)yz \eqno{6.7}$$\\
	Проинтегрируем (6.7) на $[t_1,t_2]$, получим
	$$(zy'-yz')_{t_1}^{t_2} = \int_{t_1}^{t_2}(Q_2(t) - Q_1(t))yzdt$$
	Т.к. $y(t_1)=y(t_2)=0$,
	$$y'(t_2)z(t_2) - y'(t_1)z(t_1) = \int_{t_1}^{t_2}(Q_2(t) - Q_1(t))yzdt \eqno{6.8}$$
	Исследуем знак левой части (6.8). $y'(t_1) > 0, y'(t_2) < 0$\\
	Согласно формуле Тейлора $y(t) = y(t_1) + y'(t_1)(t-t_1) + o(t-t_1)$\\
	Если $t \to t_1 + 0$, то $t - t_1 > 0, y(t) > 0 \Rightarrow y'(t_1) > 0$.
	Аналогично $y'(t_2) < 0$.\\
	$0 > y'(t_2)z(t_2) - y'(t_1)z(t_1) = \int_{t_1}^{t_2}(Q_2(t) - Q_1(t))yzdt \ge 0 \Rightarrow$ противоречие.
\end{theorem}
\qed

\begin{remark}
	Пусть на $(a,b) Q_2(t) > Q_1(t)$,тогда:
	\begin{enumerate}
		\item 
		Если решение $z(t) > 0, t\in(t_1,t_2)$, то опять приходим к противоречию.
		Получаем, что любое решение $z(t)$ уравнения (6.5) имеет хотя бы один нуль на $(t_1,t_2)$.
		\item
		Если  $z(t_1) = 0$, то следующий нуль $(z(t) = 0), t = t_\ast : t_\ast < t_2$
	\end{enumerate}
\end{remark}

\begin{theorem}[<О разделении нулей>]
	Пусть $y_1(t), y_2(t), t \in (a,b)$ - ЛНЗ уравнения.
	$y'' + Q(t)y = 0, t\in(a,b)$
	Если $t_1,t_2$ - последовательные нули решения $y_1(t)$, то решение $y_2(t)$ имеет на $(t_1,t_2)$ ровно 1 нуль.
	
	\proof
	Рассмотрим уравнения (6.4), (6.5) при условии $Q_1(t) = Q_2)t = Q(t)$.
	Покажем, что $y_2(t)$ имеет на $(t_1,t_2)$ хотя бы 1 нуль, по теореме сравнения он имеет хотя бы 1 нуль на $[t_1,t_2]$.
	Кроме того, $y_2(t) \not = 0, y_2(t_1) \not = 0$.
	(Т.е. если $y_2(t_1) = 0$, то определитель Вронского 
	$w(t) = \left|
	\begin{array}{cc}
		y_1(t) & y_2(t) \\ y_1'(t) & y_2'(t)
	\end{array}
	\right| = 0$
	при $t = t_1 \Rightarrow y_1(t), y_2(t)$ - ЛЗ $\Rightarrow$ противоречие.)
	
	Предположим противное. Пусть $\eta, \theta$ - последовательные нули $y_2(t)$ на $(t_1,t_2) : y_2(\eta) = y_2(\theta) = 0$.\\
	$y_1(\eta) \not = 0, y_1(\theta) \not = 0$ по условию. Значит по теореме сравнения $y_1$ должно иметь 0 на $(\eta,\theta)$
	$\Rightarrow$ противоречие.
\end{theorem}
\qed

\begin{corollary}[<из теорем Штурма>]
	Если $Q(t) \le 0$, то любое решение уравнения
	$$y'' + Q(t)y = 0, t\in(a,b) \eqno{6.9}$$
	имеет не более 1 нуля на (a,b).
	
	\proof
	От противного. Пусть $\exists$ решение (6.9), имеющее > 1 нуля.
	Пусть $t_1,t_2$ - его последовательные нули. Рассмотрим (6.5).
	$Q_2(t) = 0$. Т.о. $Q_2(t) \ge a_1(t)$\\
	По теореме сравнения любое решение уравнения 
	$$z'' = 0 \eqno{6.10}$$
	Будет иметь хотя бы 1 ноль на $[t_1,t_2]$, а это не так (решение $z=1,t\in(a,b)$ вообще не имеет нулей) $\Rightarrow$ противоречие.	
\end{corollary}
\qed

\begin{corollary}	
	Предположим $$0 < m^2 \le Q(t) \le M^2, t\in(a,b) \eqno{6.11}$$
	Обозначим через $\delta$ расстояние между последовательными нулями уравнения (6.9).
	Если $Q(t)$ удовлетворяет (6.11) на (a,b), то $\pi/M \le\delta\le \pi/m$.
	
	\proof
	Рассмотрим уравнения
	 $$y'' + m^2y = 0, y'' + M^2y = 0 \eqno{6.12}$$
	Их общее решение задается формулой $???WTF??$
\end{corollary}