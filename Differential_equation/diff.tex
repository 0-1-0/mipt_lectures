\documentclass{article} 
\usepackage[utf8]{inputenc} 
\usepackage[english,russian]{babel} 
\usepackage{pdfsync}
\begin{document} 

\section{Линейное однородное уравнение с переменными коэффициентами.}
В этой главе мы будем изучать свойства решений линейных уравнений 2го порядка
$$x'' + p(t)x' + q(t)x = 0, t \in (a, b) \eqno{6.1}$$
ниже предполагается, что $p(t) \in C^1 (a, b), q(t) \in C(a, b)$

\subsection{Теорема Штурма}
\underline{Лемма} 
Пусть $x(t), t \in (a,b)$ - решение уравнения (6.1);
Тогда мн-во его нулей на (содержащихся в $(a,b)$) любых отрезках $[t_1,t_2] \subset (a,b)$ конечно.

\underline{Доказательство} \\
Если $x(t), t \in (a,b)$ - решение уравнения (6.1), то $x(t) \in C^1(a,b)$ \\
Предположим, противное, т.е. решение $x(t)$ имеет бесконечное число нулей на $[t_1,t_2]$. \\
Выберем сходящуюся посл-ть нулей $\{t_k\},t_k \to t_0, k \to\infty$ 
В силу непрерывности решения $x(t)$ мы имеем
$\lim_{k\to\infty}x(t_k)= x(t_0)=0$, т.к. $\forall k \to x(t_k)=0$
$$x'(t_0)=\lim_{k\to\infty}\frac{x(t)-x(t_0)}{t - t_0} = \lim_{t_k \to t_0}\frac{x(t_k) - x(t_0)}{t_k - t_0}=0$$
$$x(t_0)=x'(t_0)=0$$
По теореме о существовании и единственности решения ЗК - единственное решение
$x(t)\equiv 0, t \in [t_1,t_2]$ - противоречие.

\underline{Следствие} \\
Если нетривиальное решение $x(t) \in (a,b)$ имеет конечное число нулей, то их можно перенумеровать.
Назовем $t_1,t_2$ последовательными нулями решения $x(t)$, если $x(t)$ не имеет нулей на $(t_1,t_2)$\\
Произведем замену функции $x(t)$  в уравнении (6.1) на ф-ию $y(t)$ согласно формуле
$$x(t) = y(t)e^{-1/2 \int_{t_0}^t p(\tau)d\tau} \eqno{6.2}$$
В результате подстановки (6.2) получим уравнение для $y(t)$:
$$y'' + Q(t)y(t) = 0, t \in (a,b) \eqno{6.3}$$
$$Q(t) = q(t) - \frac{p^{2}(t)}{4} - \frac{p'(t)}{2} \in C(a,b)$$
Рассмотрим уравнение
$$y'' + Q_1(t)y = 0 \eqno{6.4}$$
$$z'' + Q_2(t)z = 0 \eqno{6.5}$$
$$Q_2(t),Q_1(t)\in C(a,b),t\in(a,b)$$

\underline{Th. Штурма о сравнении}\\
Пусть $Q_1(t) \le Q_2(t), t\in(a,b)$, $t_1,t_2$ - два последовательных нуля решения $y(t)$ уравнения (6.4).\\
Тогда любое решение $z(t)$  уравнения (6.5) имеет хотя бы 1 ноль на $[t_1,t_2]$.
\underline{Доказательство}\\
По условию Th. решение $y(t)$ уравнения (6.4) на $(t_1,t_2)$ не меняет знак.\\
Предположим обратное: решение $z(t)$ уравнения (6.5) не имеет нулей на $[t_1,t_2]$ и будем считать, что
$z(t) > 0, t\in[t1,t2]$\\
Умножим уравнение (6.4) на $z(t)$, а уравнение  (6.5) на $y(t)$ и вычтем одно из другого, тогда 
$$zy'' - yz'' = (Q_2(t) - Q_1(t))yz \eqno{6.6}$$
Нетрудно видеть, что левая часть (6.6) преставима в виде $$zy'' - yz'' = d(zy'-yz')$$
Тогда $$d/dt(zy'-yz')=(Q_2-Q_1)yz \eqno{6.7}$$\\
Проинтегрируем (6.7) на $[t_1,t_2]$, получим
$$(zy'-yz')_{t_1}^{t_2} = \int_{t_1}^{t_2}(Q_2(t) - Q_1(t))yzdt$$
Т.к. $y(t_1)=y(t_2)=0$, то мы получим 


\end{document} ​