\section{Теория Рамсея}

\underline{Утверждение}
Среди $\forall$ 6 чел. либо некоторые 3 знакомы друг с другом, либо некоторые 3 попарно незнакомы.\\
\underline{Определение}$$s,t \in N$$
$ R(s,t)=\min\{n \in N $: при $ \forall $ раскраске ребер $K_n$ в кр. и синий цвета либо $\exists K_s \subseteq K_n $, 
у  которого все ребра кр., либо $\exists K_t \subseteq K_n $, - синие
\underline{Эквивалентное определение числа Рамсея}
$R(s,t) = \min\{n \in N \forall G=(V,E), |V|=n$, либо $\alpha(G) \ge s$, либо $\omega(G) \ge t\}$\\
$$R(1,t) = 1$$
$$R(2,t) = t$$
Существует раскраска, в которой нет ни красного $K_2$, ни синего $K_t$\\

\underline{Th.} $R(s,t) \le R(s-1,t) + R(s,t-1), \forall s,t \ge 2$
\underline{док-во:}
Пусть $n=R(s-1,t) + R(s,t-1)$. Нам нужно док., что при любой раскраске либо есть красный $K_s$, либо есть синий $K_t$.
Зафиксируем произвольную расраску ребер $K_n$ в кр. и синий цвета.
Зафиксируем произвольную вершину графа $x \in K_n$.\\
\underline{Утв.}Либо красных ребер, выходящих из $x \ge R(s,t-1)$, либо синих ребер $\ge R(s-1,t)$\\
Пусть красных ребер $\ge R(s-1,t)$. Рассмотрим множество вторых концов этих ребер.
В $K_n$ внутри $A$ проведены тоже все возможные ребра. $|A| \ge R(s-1,t)$. 
Значит в А есть либо $K_{s-1}$ красная, либо $K_t$ синяя. 
В случае, если есть красная $K_{s-1}$ можно легко достроить красную $K_s$, в случае синей $K_t$ все доказано.\\
\\

\underline{Следствие}
$R(s,t) \le C_{s+t-2}^{s-1}=C_{s+t-2}^{t-1}$
\underline{Доказательство}
По индукции.\\
База:
$R(1,t)=1, R(s,1) = 1$
$$C_{1+t-2}^{1-1}=C_{t-1}^0=1$$
Шаг:$$R(s,t) \le C_{s-1+t-2}^{s-2} + C_{s+t-1-2}^{s-1}=C_{s+t-2}^{s-1}$$
Следствие доказано.

\underline{Следствие из следствия}
$$R(s,s) \le C_{2s-2}^{s-1}$$
\underline{Следствие из сл-я из сл-я}
$$R(s,s) \le \frac{4^{s-1}}{\sqrt{\pi(s-1)}}(1 + o(1)) = (4 + o(1))^s$$

\underline{Th}Пусть n  и s таковы, что $C_n^s2^{q-C_s^2}<1$. Тогда $R(s,s) > n$
\underline{Док-во}
Нам нужно док-ть существование раскраски ребер $K_n$ в кр. и синий  цвета, при которой все $K_s \subset K_n$ неодноцв.
Рассм. случ. раскраску $\chi$ ребер у $K_n$. $C_n^2 \to $ есть различных раскрасок $2^{C_n^2}$
$$P(\chi)=2^{-C_n^2}$$
$S_1,\ldots,S_{C_n^2}$ - все $K_s$ в $K_n$\\
$A_i = \{\chi \in \Omega: S_i$ одноцв. в $\chi\}, A_i \in F$
$$P(A_i)=\frac{2*2^{C_n^2-C_s^2}}{2^{C_n^2}}=2^{1-C_s^2}$$
$$P(\cup^{C_n}_{i=1}A_i \le C_n^2*2^{1-C_s^2} < 1)$$
$$P(\neg \cup A_i) > 0$$
Теорема доказана.\\
\\

\underline{Следствие}$R(s,s) \ge \frac{s}{e\sqrt{2}}2^{s/2}$\\
\underline{Док-во}\\
$$n:=[\frac{s}{e\sqrt{2}}2^{s/2}]^n \le \frac{s}{e\sqrt{2}}2^{s/2}$$
$$C_n^s \le n^s/s! \le \frac{s^s}{e^s2^{s/2}}\frac{2^{s^2/2}}{s!}$$
$$C_n^s2^{1-C_s^2} \le \frac{s^s2^{s^2/2}}{e^s2^{3/2}s!}*2^{1-\frac{s^2-s}{2}}=\ldots <1$$
\\

Самое лучшее, что известно:\\
1)$$ R(s,s) \ge \frac{s\sqrt 2}{e}2^{s/2}(1 + o(1))$$
2)$$ R(s,s) \le 4^{s}e^{-\gamma\frac{ln^2s}{ln ln s}}$$\\

\underline{Th. (Франк, Уилсон)}\\$R(s,s) \ge(e^{1/4} + o(1))^{\frac{ln^2s}{ln ln s}} \approx e^{1/4\frac{ln^2 s}{ln ln s}}$\\
\underline{Док-во}Хотим явно указать нек. раскраску ребер $K_n, n=(e^1/4 + o(1))^\frac{ln^2 s}{ln ln s}$, при которой
все $K_s$ не одноцвенты. Это эквивалентно желанию явно указать такой граф $G=(V,E):|V|=n$ и $\alpha(G) < s, w(G) < s$.
$$V=\{\neg x = (x_1,\ldots,x_{p^3}): x+i \in {0,1}, x_1 + \ldots + x_{p^3} = p^2\}$$
$$E = \{ \{x,y\}: (x,y) \equiv o(p)\}$$

