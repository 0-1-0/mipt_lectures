\section{Теория Рамсея}

\begin{proposition}
    Среди любых 6 человек либо некоторые 3 знакомы друг с другом, либо некоторые 3 попарно незнакомы.
\end{proposition}

\begin{definition} Число Рамсея $R(s, t)$, где $s, t \in \mathbb{N}$

    $ R(s,t)=\min\{n \in N $: при $ \forall $ раскраске ребер $K_n$ в красный и синий цвета либо $\exists K_s \subseteq K_n $,
    у  которого все ребра красные, либо $\exists K_t \subseteq K_n $ у которого все ребра синие$ \} $

    $R(s,t) = \min\{n \in N \forall G=(V,E), |V|=n$, либо $\alpha(G) \ge s$, либо $\omega(G) \ge t\}$

    $R(1,t) = 1$
    $R(2,t) = t$
\end{definition}


\begin{theorem}
    \emph{ (Рамсея) }
    $R(s,t) \le R(s-1,t) + R(s,t-1), \forall s,t \ge 2$.
    \begin{proof}
        Пусть $n = R(s - 1,t) + R(s,t - 1)$. Нам нужно доказать, что при любой раскраске либо есть красный $K_s$, 
        либо есть синий $K_t$.
        Зафиксируем произвольную расраску ребер $K_n$ в красный и синий цвета.
        Зафиксируем произвольную вершину графа $x \in K_n$.
        Либо красных ребер, выходящих из $x$ не меньше $ R(s,t-1)$, либо синих ребер - не меньше $R(s-1,t)$.
        Пусть красных ребер $\ge R(s-1,t)$. Рассмотрим множество вторых концов этих ребер.
        В $K_n$ внутри $A$ проведены тоже все возможные ребра. $|A| \ge R(s-1,t)$.
        Значит в А есть либо $K_{s-1}$ красная, либо $K_t$ синяя.
        В случае, если есть красная $K_{s-1}$ можно легко достроить красную $K_s$, в случае синей $K_t$ все доказано.
        \qed
    \end{proof}
\end{theorem}

\begin{corollary}
    $R(s,t) \le C_{s+t-2}^{s-1}=C_{s+t-2}^{t-1}$
    \begin{proof}
        По индукции по $s$ и $t$.

        База:
        $R(1,t)=1, R(s,1) = 1$
        $$C_{1+t-2}^{1-1}=C_{t-1}^0=1$$

        Шаг:
        $$R(s,t) \le C_{s-1+t-2}^{s-2} + C_{s+t-1-2}^{s-1}=C_{s+t-2}^{s-1}$$
    \end{proof}
\end{corollary}

\begin{corollary}
    \emph{(из следствия)}
    $$R(s,s) \le C_{2s-2}^{s-1}$$
\end{corollary}

\begin{corollary}
    \emph{(из следствия из следствия)}
    $$R(s,s) \le \frac{4^{s-1}}{\sqrt{\pi(s-1)}}(1 + o(1)) = (4 + o(1))^s$$
\end{corollary}

\begin{theorem}
    Пусть $n$ и $s \in \mathbb{N}$ таковы, что $C_n^s 2^{q-C_s^2} < 1$. Тогда $R(s,s) > n$.
    \begin{proof}
        Нам нужно доказать существование раскраски ребер $K_n$ в красный и синий цвета, при которой все $K_s \subset K_n$ неодноцветные.
        Рассмотрим случайную раскраску $\chi$ ребер у $K_n$. Всего ребебер $C_n^2 \Rightarrow $ есть различных раскрасок $2^{C_n^2}$
        $$P(\chi)=2^{-C_n^2}$$
        $$(\Omega, \mathcal{F}, P)$$
        $$\Omega = \{\chi\}, |\Omega| = 2 ^ {C_n^2}, \mathcal{F} = 2^{\Omega}$$
        $S_1,\ldots,S_{C_n^2}$ - все $K_s$ в $K_n$\\
        $A_i = \{\chi \in \Omega: S_i$ одноцв. в $\chi\}, A_i \in F$
        $$P(A_i)=\frac{2*2^{C_n^2-C_s^2}}{2^{C_n^2}}=2^{1-C_s^2}$$
        $$P(\cup^{C_n}_{i=1}A_i \le C_n^2*2^{1-C_s^2} < 1)$$
        $$P(\neg \cup A_i) > 0$$
    \end{proof}
    \qed
\end{theorem}

\begin{corollary}
    $R(s,s) \ge \frac{s}{e\sqrt{2}}2^{s/2}$
    \begin{proof}
        $$n:=\left[ \frac{s}{e\sqrt{2}} 2^{s/2} \right]^n \le \frac{s}{e\sqrt{2}}2^{s/2}$$
        $$C_n^s \le \frac{n^s}{s!} \le \frac{s^s}{e^s2^{s/2}}\frac{2^{s^2/2}}{s!}$$
        $$C_n^s2^{1-C_s^2} \le \frac{s^s2^{s^2/2}}{e^s2^{3/2}s!}*2^{1-\frac{s^2-s}{2}}=\ldots <1$$
    \end{proof}
\end{corollary}



Самое лучшее, что известно:
\begin{enumerate}
    \item $ R(s,s) \ge \frac{s\sqrt 2}{e}2^{s/2}(1 + o(1))$
    \item $ R(s,s) \le 4^{s}e^{-\gamma\frac{ln^2s}{ln ln s}}$
\end{enumerate}

\begin{theorem}
    \emph{(Франк, Уилсон)}
    $R(s,s) \ge(e^{1/4} + o(1))^{\frac{ln^2s}{ln ln s}} \approx e^{1/4\frac{ln^2 s}{ln ln s}}$
    \begin{proof}
        Хотим явно указать некоторую раскраску ребер $K_n$, 
        $n=(e^1/4 + o(1))^\frac{ln^2 s}{ln ln s}$, 
        при которой все $K_s$ не одноцвенты. 
        Это эквивалентно желанию явно указать такой граф $G=(V,E):|V|=n$ и $\alpha(G) < s, w(G) < s$.
        $$V=\{\neg x = (x_1, \ldots, x_{p^3}): x_i \in \{0,1\}, x_1 + \ldots + x_{p^3} = p^2\}$$
        $$E = \{ \{x,y\}: (x,y) \equiv 0(p)\}$$
        $$ |V| = C_{p^3}^{p^2} $$
    \end{proof}
\end{theorem}

