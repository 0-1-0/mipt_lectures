\section{Двудольные числа Рамсея}
$b(k, k)$ - диагональное число Рамсея
$$b(k,k) \ge \frac{1}{e}k 2 ^ {k/2} (1 + o(1)) $$

\begin{theorem}
    $b(m, k) \le (1 +o(1))2^{k+1}\log{k}$
\end{theorem}

\begin{definition}
    Пусть $G \subset H$. Плотностью $G$ в $H$ называется $\frac{|E(G)|}{|E(H)|}$
\end{definition}

Вопрос: Пусть даны некоторые $m, n \in \mathbb{N}$. Расммотрим $K_{m,n}$ и возьмем некоторое $p\in(0, 1)$.
При каких условиях на $r, s \in \mathbb{N}$ мы можем гарантировать, что в $\forall$ подграфе $G$ графа $K_{m,n}$,
имеющем плотность не менее $p$, содержащимся $K_{r,s}$.

\begin{lemma}
    \emph{(псевдо-)}

    Если $m$ значительно больше, чем $r^2$, а $n$ чуть больше, чем $p^{-r}(s-1)$, то ответ на поставленный вопрос положительный.
\end{lemma}

\begin{lemma}
    Пусть $\omega(n)$ - произвольная функция: $\omega(r) \rightarrow \infty (r \rightarrow \infty)$.
    Пусть $\epsilon\in(0,1)$. Тогда $\exists \phi = \phi(r, \omega, \epsilon)$.
    
    $\phi \rightarrow 0 (r \rightarrow 0) \forall r, s, p \in (\epsilon, 1) \forall m \ge r^2 \omega(r) 
    \forall n \ge (1 + \phi)p^{-r}(s-1)$.

    $\forall G \subset K_{m, n} : плотность G \ge p$ в $G$ есть $K_{r,s}$.

    \begin{proof}
        Зафиксируем $\omega, \epsilon \in (0, 1), r, s, p, m \ge r^2 \omega(r)$.
        Нам нужно доказать, что для наличия $K_{r,s}$ в произвольном $G$ плотности $\ge p$ достаточно, чтобы 
        $n \ge (1 + \phi)p^{-r}(s-1)$ с некоторым $\phi \rightarrow 0 (r \rightarrow 0)$.
        Предположим на минуту, что в $G$ нет $K_{r,s}$. Подсчитаем, сколько в этом графе $K_{r,1}$.
        Это число $\le C_m^r(s - 1)$.

        Через $d_1, \cdots, d_n$ обозначим степени вершин в нижней грани $G$.
        Количество $K_{r,1}$ в $G$ равно $\sum_{i=1}^{n} C_{d_i}^r$. Итак, $\sum_{i=1}^{n} C_{d_i}^r \le C_m^r(s-1)$.
        Если же $\sum_{i=1}^{n} C_{d_i}^r > C_m^r(s-1)$, то $K_{r,s}$ есть в $G$.
    \end{proof}
\end{lemma}

\begin{proposition}
    $$\frac{C_{d_i}{x} + \cdots + C_{d_n}^x}{n} \ge C_{\frac{d_1+\cdots+d_n}{n}}^x$$
\end{proposition}

Следовательно, $$n C_{\frac{d_1+\cdots+d_n}{n}}^r > C_m^r(s-1)$$
$$n C_{mp}^r > C_m^r(s-1)$$
$$n > \frac{c_m^r}{c_{mp}{r}}(s-1)$$

$K^2 = o(n) \Rightarrow C_n^k ~ \frac{n^k}{k!}$.

$C_m^r \sim \frac{m^r}{r!}; p \ge \epsilon \Rightarrow mp \ge r^2w'(r) \rightarrow \infty$

$C_{mp}^{r} \sim \frac{(mp)^r}{r!}$


\begin{proof}
    (теоремы)

    $n = (1 + o(1))2^{k+1}\log k$. Нужно доказать, что $b(k,k) \le n$.
    Рассмотрим $K_{n,n}$ и зафиксируем раскраску $\chi$ его ребер в красный и синий цвета.
    Докажем, что в этой раскраске есть одноцветный $K_{k,k}$ (ка-ка-ка!).
    В соотвествии с этой раскраской покрасим вершины из нижней доли - каждую вершину красим в тот цвет, в которой окрашено не 
    менее половины выходящих из этой вершины ребер. При равенстве синих и красных красим в красный.

    $M$ - верхняя доля, $N$ - нижняя. $N_{R}, N_{B}$ - красные и синие вершины из $N$ соответственно.

    Предположим, что $|N_R| \ge |N_B|$ (второй случай аналогично. В этом случае ищем красный $K_{k,k}$).

    $m_{l} = n$, $M_{l} = M$. Индекс l - отсылка к лемме.

    $n_l = |N_R| \ge n/2$, $N_l = N_R$

    $K_{m_l, n_l}$, $G_l$ - подграф, состоящий из всех красных ребер в раскраске $\chi$.

    $p_l \ge 1/2$.

    $r_l = k - 2\log k$, $s_l = k^2 \log k$.

    Чтобы лемма была применима, надо:
    \begin{enumerate}
    \item $m_l \le r_l^2 w\omega(r_l)$
    \item $n_l \ge (1 + o(1)) p_l^{-r_l}(s_l - 1)$
    \end{enumerate}
    
    Для первого условия: $n = (1 + o(1))2^{k + 1} \log k \ge (k - 2 \log k)^2 \omega(r_l)$.

    Для второго: $2^k(\log k)(1 + o(1)) \ge (1 + o(1))2^{r_l}(k^2 \log k - 1)$,
    равное $2^k(\log k)(1 + o(1)) \ge (1 + o(1))2^{k - 2 \log k }(k^2 \log k - 1)$

    Раз оба условия выполнены, по лемме следует, что $G_l$ содержит $K_{r_l, s_l}$.

    Когда лемму будем применять второй раз, будем юзать индекс ll.

    $m_{ll} = k^2 \log k$, \
    $M_{ll} = S_l$, \
    $N_{ll} = M \\ \{$ верхние вершины $K_{r_l, s_l}$  $\}$, \
    $n_{ll} = n - (k - 2 \log k)$, \
    $G_{ll}$ - подграф, состояния из красных ребер, \
    $p_{ll} \ge \frac{1}{2} - \frac{k}{2^k}$, \
    $r_{ll} = K$, \
    $s_{ll} = 2 \log k$.

    Лемма применима, если 
    \begin{enumerate}
        \item $m_{ll} \ge r_{ll}^2 \omega(r_{ll})$,
        \item $n_{ll} \ge (1 + o(1))p_{ll}^{-r_{ll}}(s_{ll} - 1)$
    \end{enumerate}

    Проверка первого: $k^2 \log k \ge k^2 \omega(r_{ll})$.

    Проверка второго: $n - k + 2 \log k \ge (1 + o(1))(\frac{1}{2} - \frac{k}{2^k})^{-k} (2 \log k - 1)$.
    Правая часть $ \sim 2^k 2 \log k = 2^{k+1} \log k$, а левая $ \sim n $.
\end{proof}

