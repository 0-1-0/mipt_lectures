\begin{lemma}
    $$\alpha(G) \le (m+2)C_m^p$$

    $$\omega(G) \le (m+2)C_m^p$$
    
\begin{proof}

    1) Рассмотрим произвольное независимое множество вершин $W = {x_1, \cdots, x_s} : \forall i \ne j <x_i, x_j> \ne 0 (mod p) $ 

    Возьмем произвольный $x \in V \rightarrow F_x \in (FIXME) Z_p [ y_1, \cdots, y_m ]$

    $F_{\bar x}(y_1,\cdots,y_m) = F_x(y) = \prod_{i=1}^{p-1}(i - <x,y>)$

    Свойство;

    $F_x(y) = 0(mofr p) \Leftrightarrow <x,y> \ne 0 (mod p)$
    
    $F_x(y) = \sum c y_{i_1} \cdots y_{i_q}$


    Рассмторим многочлены $F_{x_1}, \cdots, F_{x_s}$ (с волной). Эти многочлены линейно независимы над $(FIXME) Z_p$

    $c_1F_{x_1} + \cdots + c_sF_{x_s} = 0$

    $\forall y \in V c_1F_{x_1}(y) + \cdots + c_sF_{x_s}(y) = 0$

    Возьмем $y = x_1$. $F_{x_1}(x_1) \ne 0 (mod p)$. $<x_1, x_1> = p^2 = 0 (mod p)$

    $F_{x_i}(x_1) = 0(mod p)$

    $c_i = 0(mod p)$

    Стало быть, многочлены линейно независимы, а значит их количество не больше размерности пространства $<F_{x_1},\ldots,F_{x_s}>$.

    В этом пространстве мономы: $1, y_i, y_iy_j, \cdots, y_{i_1},\cdots y_{i_p}$.


    $s \le \sum_{k = 0}^{p - 1} C_m^k < pC_m^p < (m + 2)C_m^p$
 

    Докажем теперь, что $\omega(G) \le (m + 2)C_m^p$.

    Рассмотрим произвольную $W = {x_1, \cdots, x_s}$ - клику в G: $\forall i \ne j <x_i, x_j> = 0 (mod p)$

    $ < x_i, x_j> \in {0, p, 2p, \ldots, p^2 - p}$, мощностью p.

    $x \in V \rightarrow F_x (FIXME) R[y_1,\cdots,y_m]$

    $F_x(y_ = (<x,y>)(<x,y> - p)(<x,y> - 2p)\cdots(<x,y>-(p^2-p))$.
    
\end{proof}
\end{lemma}

(тут было много текста...)

$R_k(s, t) = \min{\{n \in N$ при лююбой раскраске полного $k$-однородного гиперграфа на $n$ вершинах в красный и синий цвета. 
либо найдется $s$-клика, в которой все ребра красны, либо $t$ - синие$\}}$

\begin{theorem}
    $R_k(s, t) \le R_{k - 1}(R_k(s - 1, t), R_k(s, t - 1)) + 1$
    \begin{proof}
        Возьмем $n = R_{k - 1}(R_k(s - 1, t), R_k(s, t - 1)) + 1$. Рассмотрим $k$-однородные гиперграфы на $n$ вершинах. Зафиксируем раскраску $\phi$.
        Раскраска $\xi$ индуцирует раскарску $\xi'$ ребер полного $(k - 1)$ - однородного гиперграфа $H'\{x\}$
        У $H'$ не меньше $n - 1$ вершин $\Rightarrow$ по определению чисел Рамсея либо 
        $\exists$ множество $A$, $x \notin A$ : все $(k - 1)$-ребра в нем красные в $\xi'$ и 
        $|A| \ge R_k(s - 1, t)$, либо $\exists$ множество $B$, $x \notin B \cdots$
        Случаи аналогичные, рассмотрим только для $A$.

        $|A| \ge R_k(s - 1, t)$ $\Rightarrow$ по определению числа Рамсея либо $\exists A_1 \subset A$, $|A_1| = s - 1$, 
        в котором все k-ребра красные в $\xi$, либо $\exists A_2 \subset A$, $|A_2| = t, \cdots.$

        Все $k$-ребра красные в $\xi$, все $k-1$-ребра в $A_1$ красные в $\xi'$ $\Rightarrow$
        все $k$-ребра, получаемые из $(k-1)$-ребер в $A_1$, объединенных с $x$, кравсное в $\xi$.
        Других ребер не бывает.
    \end{proof}
\end{theorem}

