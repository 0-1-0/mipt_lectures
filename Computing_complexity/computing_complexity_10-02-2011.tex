	\section{Модели вычислений. Машины Тьюринга}
	Классическая машина Тьюринга:\\
	$\Sigma$ - входной алфавит, $\Gamma \subset \Sigma$ - ленточный алфавит.\\
	$\delta: Q\times\Gamma \to Q\times\Gamma\times\{L,R,N\}$ - программа.\\
	$q_1$ - начальное состояние,  $q_a, q_r$ - принимающее и отвергающее состояния.\\
	
	Варианты машин:
	\begin{enumerate}
		\item $\delta: Q\times\Gamma \to Q\times\Gamma\times\{L,R\}$
		\item Лента, бесконечная лишь с одной стороны
		\item Уменьшение алфавита $\Sigma$
		\item  Многоленточные машины $\delta:Q\times\Gamma^k \to Q\times\Gamma^k\times \{L,R,N\}^k$
	\end{enumerate} 
	\underline{Тезис Черча-Тьюринга}\\Любой алгоритм можно реализовать на МТ.\\
	\underline{Усиленный:}\\Любую вычислительную систему можно смоделировать на МТ с не более чем полиномиальным временем.\\
	\emph{Конфигурация} - набор $AqaB$, 
	где $q$ - текущее состояние, $a$ - текущий символ, $A$ - слово слева от $a$, $B$ - слово справа.\\ Кроме $AaB$ на ленте только пробелы \\
	\emph{Протокол} - последовательность конфигураций в процессе работы.\\
	\emph{Универсальная МТ}: $U(p,x) = M_p(x)$\\
	\emph{Язык} $L \subset \{0,1\}^*$\\
	
	
	\underline{Класс P} $ = \cup_{k=1}^{\infty}DTIME(n^k)$,	$L \in DTIME(t(n))$, если $\exists$ МТ $M$:
	\begin{enumerate}
		\item Если $x\in L$, то  $M(x)=1$
		\item Если $x \not\in L$, то $M(x)=0$
		\item $\forall x \exists c$, если $|x|=n$, то  $M(x)$ работает $\le ct(n)$ шагов.\\
	\end{enumerate}
	
	
	\underline{Класс NP}: $L\in NP$, если $\exists$ алгоритм $V(..):$
	\begin{enumerate}
		\item $x\in L \to \exists s:|s| \le p(|x|), V(x,s)=1$
		\item $x\not\in L \to \forall s:|s| \le p(|x|), V(x,s)=0$
		\item $\forall x \forall s |s| \le p(|x|), V$ работает не более чем за $q{|x|}$ шагов.
	\end{enumerate}
	\underline{Th.}$P \subset NP$\\
	\underline{Док-во:}$V(x,s)=M(x)$\\
	
	
	\section{Недетерминированные МТ}
	Может быть несколько команд с одной и той же Л.И.$\delta:Q\times\Gamma \Rightarrow Q\times\Gamma\times\{L,R,N\}$\\
	Если несколько вариантов, вычисления разделяются на ветви.\\
	Если на хотя бы на одной ветви $q_a$ - ответ 1.\\
	Если везде $q_r$ - ответ 0.\\
	Если есть бесконечная ветвь - ответа нет.\\
	
	\underline{NTIME(t(n))} - класс языков L: $\exists$ НМТ $M$:
	 \begin{enumerate}
	 	\item $x \in L \to M(x)=1$
		\item $X \not\in L \to M(x)=0$
		\item $\exists c \forall x$ любая ветвь $M(x)$ работает не более чем за $ct(|x|)$ шагов.
	 \end{enumerate}
	
	\underline{Класс NP} $ = \cup_{k=1}^{\infty}NTIME(n^k)$