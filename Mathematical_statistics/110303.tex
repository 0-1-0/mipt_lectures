Пусть $\hat{\theta_n}(x)$ приближающая помследовательность оценок. Как построить хорошую оценку для $\tau(\theta)$? 
\begin{lemma} утверждение
    Пусть $\hat{\theta_n}(x)$ - сильносостоятельная оценка для $\theta$. Если $\tau(\theta)$ непрерывная на $\Theta \subset R$, то $\tau(\hat{\theta(x)})$.
    Состоятельная оценка $\tau(\theta)$. Следует из теоремы о наследовании сходимости. 
\end{lemma}

\begin{lemma} Лемма о наследовании ассимптотической нормальности.

    Пусть $\hat{\theta_n(x)}$ - ассимптотически нормальная оценка $\theta$ с ассимптотической дисперсией $\sigma^2(\theta)$.

    Пусть $\tau(\hat{\theta(x)})$ - дифференцируема в $\forall \theta \in \Theta \subset R$.
    Тогда $t(\hat{\theta}(x))$ - с асиимптотической дисперсией $\sigma^2(\theta)(\tau(\theta))$.

    Для $\forall \theta \in \Theta$  выполняется $\sqrt{n}(\hat{\theta}(x) - \theta) \rightarrow(d) Normal(0, \sigma^2(\theta))$.
    
    Воспользуемся следствием из леммы Слуцкого

    SKIP


    $\xi_n = \sqrt{n}(\hat{\theta}(x)- \theta)$
    $\xi = normal(0, \sigma^2(\theta)$

    $h(x) = \tau(x)$

    $a = \theta b = \frac{1}{\sqrt{n}}$

     SKIP
\end{lemma}

\begin{example}
    Пусть $x_1, \cdots, x_n$ - выборка из $Exp(\theta)$. Найти ассимптотически нормальную оценку для $\theta$.
    \begin{proof} решение

        Плотность $x_i$ есть $p_{\theta}(x) = \theta e^{-\theta x} I(x > 0)$
        Тогда $E_{\theta}x_1 = \frac{1}{\theta}, D_{\theta}x_1 = \frac{1}{\theta^2}$
        Следовательно
        SKIP


        $\tau(\bar{x})$ - ассимптотическая нормальная оценка для $\tau(\frac{1}{\theta})$ с ассимптотической дисперсией 
        $\frac{1}{\theta ^2} (\tau' (\frac{1}{\theta})) ^ 2 = \theta^2$
    \end{proof}
\end{example}

\section{Методы нахождения оценок}
Примеры подстановок
Пусть у нас имеется некоторая выборка $x_1, \cdots, x_n$ - выборка из $P \in \{P_{\theta}, \theta \in \Theta\}$

Пусть $P \in \{P_{\theta}, \theta \in \Theta\}$ такова, что для некоторого функционала G выполнено

$\forall \theta \in \Theta: \theta = G(P_{\theta})$

Тогда $\theta*(x_1, \cdots, x_n) = G(P_n^*)$ - называется оценкой параметра $\theta$ по методу подстановки,
где $P_n^*$ - эмпирическая оценка плотности по выборке $x_1, \cdots, x_n$.

\section{Метод моментов}
SKIP

Пусть $x_1, \cdots, x_n$ - таковы, что вектор

SKIP


\begin{theorem} Состоятельность оценки метода моментов
    Пусть $\theta^*(x_1, \cdots, x_n)$ - оценка $\theta \in \Theta \subset R^k$ с приближенными SKIP
\end{theorem}
